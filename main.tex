% ****** Start of file apssamp.tex ******
%
%   This file is part of the APS files in the REVTeX 4.1 distribution.
%   Version 4.1r of REVTeX, August 2010
%
%   Copyright (c) 2009, 2010 The American Physical Society.
%
%   See the REVTeX 4 README file for restrictions and more information.
%
% TeX'ing this file requires that you have AMS-LaTeX 2.0 installed
% as well as the rest of the prerequisites for REVTeX 4.1
%
% See the REVTeX 4 README file
% It also requires running BibTeX. The commands are as follows:
%
%  1)  latex apssamp.tex
%  2)  bibtex apssamp
%  3)  latex apssamp.tex
%  4)  latex apssamp.tex
%
\documentclass[%
 reprint,
%superscriptaddress,
%groupedaddress,
%unsortedaddress,
%runinaddress,
%frontmatterverbose, 
%preprint,
%showpacs,preprintnumbers,
%nofootinbib,
%nobibnotes,
%bibnotes,
 amsmath,amssymb,
 aps,
%pra,
%prb,
%rmp,
%prstab,
%prstper,
%floatfix,
]{revtex4-1}
\usepackage[
backend=biber,
style=numeric,
sorting=none
]{biblatex}
\addbibresource{bib.bib} %Imports bibliography file
\usepackage{graphicx}% Include figure files
\usepackage{dcolumn}% Align table columns on decimal point
\usepackage{bm}% bold math
%\usepackage{hyperref}% add hypertext capabilities
%\usepackage[mathlines]{lineno}% Enable numbering of text and display math
%\linenumbers\relax % Commence numbering lines

%\usepackage[showframe,%Uncomment any one of the following lines to test 
%%scale=0.7, marginratio={1:1, 2:3}, ignoreall,% default settings
%%text={7in,10in},centering,
%%margin=1.5in,
%%total={6.5in,8.75in}, top=1.2in, left=0.9in, includefoot,
%%height=10in,a5paper,hmargin={3cm,0.8in},
%]{geometry}

\begin{document}

%\preprint{APS/123-QED}

\title{Exchange Interactions in Metals and Alloys}% Force line breaks with \\

\author{Nikolaos Palamidas\\Supervisor: Mark van Schilfgaarde}
 \altaffiliation{Physics Department, King's College London}%Lines break automatically or can be forced with \\

\date{March 2017}
\begin{abstract}
Self consistent electronic structure calculations such as density functional theory particularly within the context of the local density approximation, were employed to investigate the electrical and magnetic properties of Permalloy(Iron-Nickel alloy) and its variants. Specifically, its application to spintronic transistors is investigated, with emphasis on selection of the best Permalloy alloy ratio to amplify desired properties, and suppress some others.

\end{abstract}

\pacs{Valid PACS appear here}% PACS, the Physics and Astronomy
                             % Classification Scheme.
%\keywords{Suggested keywords}%Use showkeys class option if keyword
                              %display desired
\maketitle

%\tableofcontents

\section{\label{sec:level1}Introduction}

Moore's law describes the exponential relationship between years since 1971 and the amount of transistors technology companies could fit in an area of microchip. While technology has improved exponentially and we continue to grow, Moore's law is failing rather rapidly. We run into quantum mechanical barriers when making transistors so small. New technologies must be developed to adapt to faster processing, especially in the wake of quantum computing. One hope in this is superconducting magnetoresistive random access memory or JMRAM. It uses the idea of Josephson junctions to store logic bit states by the physical phenomenon of magnetoresistance. While very expensive, time consuming, and labour intensive to physically create these JMRAM devices for testing, computational simulation does not require much funding, time or physical labour. It is essential for the progression of these technologies to understand the solid-state physics behind these devices. Many significant and tested methods are employed, including self consistent density functional theory\cite{inhom}, by use of the local density approximation. A physical basis set must also be constructed, so Linear Muffin Tin orbitals (LMTOs) \cite{andersen} in the atomic sphere approximation(ASA) \cite{gunn} have been employed extensively to accurately model alloys used in the construction of JMRAM technologies.

\section{Density Functional Theory}

First introduced by Hohenberg and Kohn in 1964 \cite{inhom}, the method of Density Functional Theory(DFT) has been employed in many fields of physics as a means of calculating ground state properties of electronic systems, and has proved extremely successful. It begins by ascertaining that energy is a unique functional of the electron density n(r). An effective non interacting potential is introduced, with all interaction energies absorbed in the exchange-correlation functional $E_X_C[n(r)]$. By variational and self-consistent computational methods, one is able to compute the ground state energy $E_0$ and electron density $n_0(r)$. The $E_X_C[n(r)]$ in these metal alloy environments is decently represented by that of the local spin density approximation (LSDA) \cite{exc,inhom}. One solves computationally the Kohn-Sham equation (1).

\begin{equation}
    H_s \psi_i (r) = (-\frac{\hbar^2}{2m}\nabla^2 +V_e_f_f(r) ) \psi_i (r) = \epsilon_i \psi_i (r)
\end{equation}

By variation of the electron density by many computational iteration, one may obtain the minimal solution to (1) and obtain the ground state energy, a quantity so essential to the understanding of electronic systems\cite{martin}.

\section{Linear Muffin Tin Orbitals}

The theory of LMTOs, initially developed by O.K. Andersen\cite{andersen}, has proved to be a very successful model to create approximate wavefunctions to describe electrons in solids. By approximation of a flat potential with a lattice spherically symmetric potentials on a lattice of sites. Solution of the Schrodinger equation to this 3D potential, yields appropriate Bessel Function solutions. By finding a linearisation energy $E_\nu$, one may produce energy independent muffin tin orbitals.

\begin{equation}
    
\end{equation}




\subsection{\label{sec:level2}Second-level heading: Formatting}

This file may be formatted in either the \texttt{preprint} or
\texttt{reprint} style. \texttt{reprint} format mimics final journal output. 
Either format may be used for submission purposes. \texttt{letter} sized paper should
be used when submitting to APS journals.

\subsubsection{Wide text (A level-3 head)}
The \texttt{widetext} environment will make the text the width of the
full page, as on page~\pageref{eq:wideeq}. (Note the use the
\verb+\pageref{#1}+ command to refer to the page number.) 
\paragraph{Note (Fourth-level head is run in)}
The width-changing commands only take effect in two-column formatting. 
There is no effect if text is in a single column.

\subsection{\label{sec:citeref}Citations and References}
A citation in text uses the command \verb+\cite{#1}+ or
\verb+\onlinecite{#1}+ and refers to an entry in the bibliography. 
An entry in the bibliography is a reference to another document.

\subsubsection{Citations}
Because REV\TeX\ uses the \verb+natbib+ package of Patrick Daly, 
the entire repertoire of commands in that package are available for your document;
see the \verb+natbib+ documentation for further details. Please note that
REV\TeX\ requires version 8.31a or later of \verb+natbib+.

\paragraph{Syntax}
The argument of \verb+\cite+ may be a single \emph{key}, 
or may consist of a comma-separated list of keys.
The citation \emph{key} may contain 
letters, numbers, the dash (-) character, or the period (.) character. 
New with natbib 8.3 is an extension to the syntax that allows for 
a star (*) form and two optional arguments on the citation key itself.
The syntax of the \verb+\cite+ command is thus (informally stated)
\begin{quotation}\flushleft\leftskip1em
\verb+\cite+ \verb+{+ \emph{key} \verb+}+, or\\
\verb+\cite+ \verb+{+ \emph{optarg+key} \verb+}+, or\\
\verb+\cite+ \verb+{+ \emph{optarg+key} \verb+,+ \emph{optarg+key}\ldots \verb+}+,
\end{quotation}\noindent
where \emph{optarg+key} signifies 
\begin{quotation}\flushleft\leftskip1em
\emph{key}, or\\
\texttt{*}\emph{key}, or\\
\texttt{[}\emph{pre}\texttt{]}\emph{key}, or\\
\texttt{[}\emph{pre}\texttt{]}\texttt{[}\emph{post}\texttt{]}\emph{key}, or even\\
\texttt{*}\texttt{[}\emph{pre}\texttt{]}\texttt{[}\emph{post}\texttt{]}\emph{key}.
\end{quotation}\noindent
where \emph{pre} and \emph{post} is whatever text you wish to place 
at the beginning and end, respectively, of the bibliographic reference
(see Ref.~[\onlinecite{witten2001}] and the two under Ref.~[\onlinecite{feyn54}]).
(Keep in mind that no automatic space or punctuation is applied.)
It is highly recommended that you put the entire \emph{pre} or \emph{post} portion 
within its own set of braces, for example: 
\verb+\cite+ \verb+{+ \texttt{[} \verb+{+\emph{text}\verb+}+\texttt{]}\emph{key}\verb+}+.
The extra set of braces will keep \LaTeX\ out of trouble if your \emph{text} contains the comma (,) character.

The star (*) modifier to the \emph{key} signifies that the reference is to be 
merged with the previous reference into a single bibliographic entry, 
a common idiom in APS and AIP articles (see below, Ref.~[\onlinecite{epr}]). 
When references are merged in this way, they are separated by a semicolon instead of 
the period (full stop) that would otherwise appear.

\paragraph{Eliding repeated information}
When a reference is merged, some of its fields may be elided: for example, 
when the author matches that of the previous reference, it is omitted. 
If both author and journal match, both are omitted.
If the journal matches, but the author does not, the journal is replaced by \emph{ibid.},
as exemplified by Ref.~[\onlinecite{epr}]. 
These rules embody common editorial practice in APS and AIP journals and will only
be in effect if the markup features of the APS and AIP Bib\TeX\ styles is employed.

\paragraph{The options of the cite command itself}
Please note that optional arguments to the \emph{key} change the reference in the bibliography, 
not the citation in the body of the document. 
For the latter, use the optional arguments of the \verb+\cite+ command itself:
\verb+\cite+ \texttt{*}\allowbreak
\texttt{[}\emph{pre-cite}\texttt{]}\allowbreak
\texttt{[}\emph{post-cite}\texttt{]}\allowbreak
\verb+{+\emph{key-list}\verb+}+.
\printbibliography
\end{document}
%
% ****** End of file apssamp.tex ******
